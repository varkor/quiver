\documentclass[12pt,a4paper]{article}

\usepackage[margin=2.5cm]{geometry}
\usepackage{microtype}

% For the date.
\usepackage[UKenglish]{isodate}

% For URLs.
\usepackage{hyperref}[pdfusetitle]
\hypersetup{colorlinks,urlcolor=blue}
\urlstyle{rm}

% For tables.
\usepackage{tabularray}

% For side-by-side examples.
\usepackage[skins,listings]{tcolorbox}
\tcbuselibrary{minted}
\newtcblisting{tikzexample}{
    sidebyside,
    center lower,
    bicolor,
    colbacklower=white,
    sharp corners,
    frame engine=empty
}

% The package in question.
\usepackage{quiver}

\newcommand{\quiver}{\textsf{quiver}}
\newcommand{\tcd}{\textsf{tikz-cd}}
\newcommand{\TikZ}{{\rm Ti{\it k}Z}}

\title{\textsf{quiver} \\ \small A modern commutative diagram editor}
\author{\normalsize \textsf{@varkor}}
\date{\cleanlookdateon\today}

\begin{document}

\maketitle

\begin{abstract}
    The \textsf{quiver} package provides compatibility with the \textsf{quiver} commutative diagram editor.
\end{abstract}

\section{Usage}

The \quiver{} package is intended to be used in conjunction with the \url{https://q.uiver.app/} commutative diagram editor. It functions as a wrapper around the \tcd{} package that provides some extra \TikZ{} styles, which are used in diagrams exported by the editor. However, these styles may be used in isolation of the editor.

To use the \quiver{} package, add the following to the preamble of your document.

\begin{verbatim}
\usepackage{quiver}
\end{verbatim}

\section{\TikZ{} styles}

The \quiver{} package provides several styles that are useful in the rendering of commutative diagrams.

\subsection{The \texttt{curve} style}

The typical method of curving arrows in \tcd{} commutative diagrams is to use the \TikZ{} \texttt{bend left} and \texttt{bend right} styles, together with the \texttt{in} and \texttt{out} parameters. These styles facilitate the bending of paths with respect to an angle. However, it is often more convenient to bend a path with respect to a distance (namely, the height of the apex of the curve). This may be achieved with the \texttt{curve} style\footnotemark{}.
\footnotetext{The \texttt{curve} style is due to Andr\'eC (\url{https://tex.stackexchange.com/users/138900/andr\%C3\%A9c}).}

\begin{tikzexample}
\begin{tikzcd}
	a \ar["f", rr, curve={height=-24pt}] && b
\end{tikzcd}
\end{tikzexample}

The \texttt{curve} style takes two (optional) named parameters.

\begin{itemize}
    \item \texttt{height} specifies the height of the apex of the curve. It is a signed distance (default \texttt{0}).
    \item \texttt{pos} specifies the position of the control point for the curve. It is a value between \texttt{0.0} and \texttt{1.0} (default \texttt{0.35}).
\end{itemize}

\begin{tikzexample}
\begin{tikzcd}
	a \ar["f", rr, curve={height=-24pt, pos=0.2}] && b
\end{tikzcd}
\end{tikzexample}

\begin{tikzexample}
\begin{tikzcd}
	a \ar["f", rr, curve={height=-24pt, pos=0.5}] && b
\end{tikzcd}
\end{tikzexample}

\begin{tikzexample}
\begin{tikzcd}
	a \ar["f", rr, curve={height=-24pt, pos=0.8}] && b
\end{tikzcd}
\end{tikzexample}

\subsection{Arrowhead and tail styles}

The \tcd{} package provides several arrowhead and tail styles. The \quiver{} package provides the following additional styles.

\begin{center}
    \begin{tblr}{cc}
        {\bf Style} & {\bf Appearance} \\
        \texttt{tail reversed} &
        % https://q.uiver.app/?q=WzAsMixbMCwwXSxbMSwwXSxbMCwxLCIiLDAseyJzdHlsZSI6eyJ0YWlsIjp7Im5hbWUiOiJhcnJvd2hlYWQifX19XV0=
        \begin{tikzcd}[ampersand replacement=\&]
        	{} \& {}
        	\arrow[tail reversed, from=1-1, to=1-2]
        \end{tikzcd}
        \\
        \texttt{2tail} &
        % https://q.uiver.app/?q=WzAsMixbMCwwXSxbMSwwXSxbMCwxLCIiLDAseyJsZXZlbCI6Miwic3R5bGUiOnsidGFpbCI6eyJuYW1lIjoibW9ubyJ9fX1dXQ==
        \begin{tikzcd}[ampersand replacement=\&]
        	{} \& {}
        	\arrow[Rightarrow, 2tail, from=1-1, to=1-2]
        \end{tikzcd}
        \\
        \texttt{2tail reversed} &
        % https://q.uiver.app/?q=WzAsMixbMCwwXSxbMSwwXSxbMCwxLCIiLDAseyJsZXZlbCI6Miwic3R5bGUiOnsidGFpbCI6eyJuYW1lIjoiYXJyb3doZWFkIn19fV1d
        \begin{tikzcd}[ampersand replacement=\&]
        	{} \& {}
        	\arrow[Rightarrow, 2tail reversed, from=1-1, to=1-2]
        \end{tikzcd}
        \\
        \texttt{no body} &
        % https://q.uiver.app/?q=WzAsMixbMCwwXSxbMSwwXSxbMCwxLCIiLDAseyJzdHlsZSI6eyJib2R5Ijp7Im5hbWUiOiJub25lIn19fV1d
        \begin{tikzcd}[ampersand replacement=\&]
        	{} \& {}
        	\arrow[no body, from=1-1, to=1-2]
        \end{tikzcd}
    \end{tblr}
\end{center}

\section{Version history}

The \quiver{} version number is tied to the editor version number. This is reflected in the history below: not every editor update corresponds to a package update.

\begin{center}
    \begin{tblr}{colspec={ccX[c]},row{even}=black!5}
        {\bf Version} & {\bf Date} & {\bf Release notes} \\
        1.0.0 & 2020-11-10 & First public version. Imports necessary packages and defines the \texttt{curve} style. \\
        1.0.1 & 2020-11-27 & Added the \texttt{tail reversed}, \texttt{2tail}, and \texttt{2tail reversed} styles. \\
        1.1.0 & 2020-12-18 & Added a dependency. \\
        1.2.0 & 2021-01-11 & Added the \texttt{no body} style.
    \end{tblr}
\end{center}

\end{document}
